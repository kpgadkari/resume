% LaTeX resume using res.cls
\documentclass[line,margin]{res}
%\usepackage{helvetica} % uses helvetica postscript font (download helvetica.sty)
%\usepackage{newcent}   % uses new century schoolbook postscript font
\usepackage{url}
%\setlength{\parskip}{0pt}
\setlength{\parsep}{0pt}
%\setlength{\headsep}{0pt}
%\setlength{\topskip}{0pt}
%\setlength{\topmargin}{0pt}
%\setlength{\topsep}{0pt}
%\setlength{\partopsep}{0pt}

\begin{document}

\name{Kaustubh Gadkari}
% \address used twice to have two lines of address
\address{37887 Benchmark Ct., Fremont CA 94536}
\address{Email: kaustubh.gadkari@gmail.com \\
Phone: +1-970-261-1699}
%\address{\url{http://www.netsec.colostate.edu/~kaustubh}}


\begin{resume}

\section{OBJECTIVE}
 An full time position in the field of computer networks with special
               interests in network measurements, scalable forwarding
               techniques and routing.


\section{EDUCATION}
                {\sl Ph.D.} Computer Science \hfill Spring 2011 - Present\\
                    Colorado State University, Fort Collins, CO 80523\\
                    Network Security Group\\
                    Co-advisers: Dr. Christos Papadopoulos and Dr. Daniel Massey

                {\sl Master's Degree.} Computer Science \hfill Fall 2010\\
                    Colorado State University, Fort Collins, CO 80523

                {\sl Bachelor of Engineering.} Information Technology \hfill Fall 2006\\
                	  University of Pune, Pune, India

\section{COURSES \& PROJECTS}
					{\sl BGPMon} \hfill Spring 2012 - Present
				\begin{itemize} \itemsep -2pt
							\item Implemented a system in Perl to automatically archive BGPmon XML streams.
							\item Code is available via CPAN.
						\end{itemize}

					{\sl Computer Security} \hfill Fall 2007
						\begin{itemize} \itemsep -2pt
						\item Did a comprehensive study of Mandatory Access Control mechanisms, with a focus on SELinux.
						\end{itemize}

					{\sl Advanced Topics in Computer Security} \hfill Fall 2007
						\begin{itemize} \itemsep -2pt
						\item Presented and critiqued computer security themed papers, with a focus on privacy models.
						\end{itemize}


					{\sl Advanced Networking} \hfill Fall 2007
			 		\begin{itemize} \itemsep -2pt
                        \item Created an information distribution system that later inspired parts of BGPMon.
                        \item Researched the benefits of BGP prefix caching.
                    \end{itemize}

					{\sl Advanced Topics In Networking} \hfill Spring 2008
					\begin{itemize} \itemsep -2pt
						\item Implemented testbed using multiple machines and Cisco 2600 routers.
						\item Measured packet rates required to cause the router to drop packets and deny service to hosts.
					\end{itemize}


					{\sl Internet Engineering} \hfill Spring 2009
                    \begin{itemize}
                        \item Implemented reliable transport and shortest path routing protocols over UDP on PlanetLab.
                        \item Developed a CHORD-based P2P network.
                    \end{itemize}

					{\sl Image Computation} \hfill Spring 2010
                    \begin{itemize} \itemsep -2pt
                        \item Implemented a ray tracer, video stabilizer, and object detection software.
                    \end{itemize}



\section{EXPERIENCE}
	{\sl Network Data Scientist} \hfill July 2015 - Present\\
	ThousandEyes, Inc.
	\begin{itemize}
		\item Analysis of network data to analyze network failures and perform
			root-cause analysis of failures.
		\end{itemize}
	{\sl Internship} \hfill January 2015 - July 2015\\
	Ooyala, Inc.
	\begin{itemize}
		\item Worked on design of data center networks for cloud-based video
			delivery.
	\end{itemize}
	{\sl Internship} \hfill August 2014 - January 2015\\
	Naddive LLC, Fort Collins CO 80521
	\begin{itemize}
		\item Worked on packet inspection techniques to improve live-event video
			delivery over wirelss networks.
	\end{itemize}
    {\sl Internship} \hfill Summer 2011\\
      Palo Alto Research Center, Palo Alto CA 94304 \\
      \begin{itemize}
      \item Worked on performance measurements of ``Content Centric
		  Networking.''(\url{http://www.ccnx.org}).
      \item Performance measurements were carried out on PlanetLab (\url{http://www.planet-lab.org/}) and the Open Network
      Laboratory (\url{https://onl.wustl.edu/}).
      \item Compared performance of CCN with TCP as well as CDNs such as CoralCDN.
      \end{itemize}

    {\sl Graduate Research Assistant} \hfill Fall 2009 - Present\\
        Colorado State University, Fort Collins CO 80523
        \begin{itemize}  \itemsep -2pt
				\item {\bf Interests}: Network measurements, routing with reduced tables, content-centric networking.
            \item Researching new designs for reduced forwarding tables.
            \item Studied the use of fingerprinting custom TCP/IP stacks to detect botnets.
            \item Studied the characteristics of TCP resets.
        \end{itemize}

   {\sl Graduate Teaching Assistant} \hfill Fall 2008 - Fall 2009\\
        Colorado State University, Fort Collins CO 80523\\
        Have assisted the following courses:
        \begin{itemize}  \itemsep -2pt
            \item CS 457 - Computer Networking and the Internet.
            \item CS 475 - Parallel Programming
            \item CS 155 - Introduction to Unix
            \item CS 156 - Introduction to C, part 1
            \item CS 157 - Introduction to C, part 2
         \end{itemize}

    {\sl NDN Hub Operator} \hfill {Fall 2010 - Spring 2013}\\
     	Colorado State University, Fort Collins CO 80523
	\begin{itemize} \itemsep -2pt
		\item Setup and maintain one of the two central hubs of the Named Data
		Networking Testbed.
	\end{itemize}

    {\sl Network Security Lab System Administrator} \hfill Fall 2006 - Present\\
        Colorado State University, Fort Collins CO 80523
        \begin{itemize} \itemsep -2pt
            \item Setup and currently maintain the Network Security Group Lab,
            consisting of 30 rack mounted servers and 5 desktop computers
            running Ubuntu Linux/FreeBSD and a Cisco router.
            \item Setup and maintain data capture points at a regional ISP, using
            specialized packet capture hardware from Endace.
        \end{itemize}

    {\sl Visiting Lecturer} \hfill 2006\\
        Symbiosis Institute of Computer Studies and Research, Pune, India
        \begin{itemize} \itemsep -2pt
            \item Taught Linux system administration to students pursuing the Master
            of Science (Computer Applications) degree.
        \end{itemize}

    {\sl Undergraduate Research} \hfill Fall 2005 - Fall 2006\\
        SONIX Systems, Pune, India
        \begin{itemize} \itemsep -2pt
            \item Worked on a project to develop a process infrastructure for the Linux kernel to allow
            transparent process migration across cluster nodes.
        \end{itemize}

\section{PUBLICATIONS}
            \begin{enumerate}
              \item {\sl Pragmatic Router FIB Caching}, Kaustubh Gadkari, M. Lawrence Weikum, Dan Massey and
				  Christos Papadopoulos, Networking 2015, May 2015

		  	\item {\sl A Fresh Look At Scalable Forwarding Through Router FIB Caching}, Kaustubh Gadkari, Dan Massey and
              Christos Papadopoulos, NANOG 57, February 2013

              \item {\sl Dynamics of Prefix Usage at an Edge Router}, Kaustubh Gadkari, Dan Massey and Christos Papadopoulos,
              ${12^{th}}$ Passive and Active Measurements Conference (PAM 2011), March 2011

         	  \item {\sl Fingerprinting Custom Botnet Protocol Stacks}, Steve DiBenedetto, Kaustubh Gadkari, Nicholas Diel,
              Andrea Steiner, Dan Massey and Christos Papadopoulos, Workshop on Secure Network Protocols (NPSec 2010)
              (in conjunction with ICNP 2010), October 2010

              \item {\sl Dynamics of RIB Usage at an Edge Router} (Poster), Kaustubh Gadkari, Steve DiBenedetto,
               Dan Massey and Christos Papadopoulos, ${18^{th}}$ International Conference on Network Protocols (ICNP 2010), October 2010

               \item {\sl Characterizing TCP Resets in Established Connections}, Nicholas Diel, Kaustubh Gadkari, Steve DiBenedetto,
              Andrea Steiner and Christos Papadopoulos,
              Technical Report {\emph {CS-08-102}}.
             \end{enumerate}

\section{EXPERTISE AND SKILLS}
	\begin{itemize}
		\item Programming Languages: C, C++, Perl and Python.
		\item Operating Systems: Windows, GNU/Linux and OS X.
		\item Other software: Microsoft Office, R, LaTeX.
		\item Expertise in network data analysis and working with terabytes of packet-level and flow-level network traces.
		\item Extensive experience with Perl and Python and automating data
			visualization with GNUplot and matplotlib.
	\end{itemize}

\section{SERVICES}
	\begin{itemize}
                \item Web and Publications Chair, $4^{th}$ Workshop on Secure Network Protocols, 2008
                \item Ad hoc reviewer for conferences/workshops like INFOCOM 2013, INFOCOM 2010, GLOBECOM 2011 and NPSec 2008.
                \item Helped organize CANVAS 2008, which was a computer security vulnerability assessment game for undergraduate students.
         \end{itemize}



\section{ACADEMIC MEMBERSHIPS}
	\begin{itemize}
		\item ACM Student Member
		\item IEEE Student Member
		\item Internet Society (ISOC)
		\item Colorado Chapter of the ISOC
	\end{itemize}


\end{resume}
\end{document}







